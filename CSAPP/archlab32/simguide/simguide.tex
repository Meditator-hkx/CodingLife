\documentclass[11pt]{article}

\usepackage{times}
\usepackage{alltt}
\usepackage[dvips]{graphicx} 
\usepackage{color}

\includeonly{}

\input{/afs/cs/project/ics2/decls-handout}
\input{/afs/cs/project/ics2/decls-common}
\input{/afs/cs/project/ics2/data/data-defs}
\input{/afs/cs/project/ics2/asm/asm-defs}
\input{/afs/cs/project/ics2/arch/arch-defs}

\begin{document}
\bibliographystyle{plain}

\title{CS:APP2e Guide to Y86 Processor Simulators%
\thanks{Copyright \copyright{} 2002, 2010, R. E. Bryant, D. R.
O'Hallaron.  All rights reserved.}\\
\vspace*{0.75in}
\centerline{\includegraphics*[scale=.36]{/afs/cs/project/ics2/arch/pipe-full.eps}}
}

\author{Randal E. Bryant\\ David R. O'Hallaron}

\maketitle

\include{/afs/cs/project/ics2/book/book}

This document describes the processor simulators that accompany the
presentation of the Y86 processor architectures in Chapter
\ref{chap:arch} of {\em Computer Systems: A Programmer's Perspective,
  Second Edition}.  
These simulators model three different processor designs: SEQ, SEQ+,
and PIPE\@.

\section{Installing}

The code for the simulator is distributed as a tar format file named
 \texttt{sim.tar}.  You can get a copy of this file from the CS:APP2e
 Web site (\verb@csapp.cs.cmu.edu@).


With the tar file in the directory you want to install the code, you
should be able to do the following:
\begin{tty}
\unixprompt{\em tar xf sim.tar}
\unixprompt{\em cd sim}
\unixprompt{\em make clean}
\unixprompt{\em make}
\end{tty}

By default, this generates GUI (graphic user interface) versions of
the simulators, which require that you have Tcl/Tk installed on your
system. If not, then you have the option to install TTY-only
versions that emit their output as ASCII text on
stdout. See file {\tt README} for a description of
how to generate the GUI and TTY versions.

The directory \texttt{sim} contains the following subdirectories:

\begin{description}
\item[\texttt{misc}]

Source code files for utilities such as \textsc{yas} (the Y86
assembler), \textsc{yis} (the Y86 instruction set simulator), and
\textsc{hcl2c} (HCL to C translator).  
It also contains the {\tt isa.c} source file that is  used by all of the 
processor simulators.

\item[\texttt{seq}]

Source code for the SEQ and SEQ+ simulators.  Contains the HCL file
for homework problems \ref{prob:arch:seq-iaddl} and
\ref{prob:arch:seq-leave}.  
See file \texttt{README} for instructions 
on compiling the different versions of the simulator.


\item[\texttt{pipe}]

Source code for the PIPE simulator.
Contains the HCL files for homework problems
\ref{prob:arch:pipe-iaddl}--\ref{prob:arch:pipe-one-write-port}.  See
file \texttt{README} for instructions on compiling the different
versions of the simulator.

\item[\texttt{y86-code}]

Y86 assembly code for many of the example programs shown in the
chapter.  You can automatically test your modified simulators on these
benchmark programs.  See file \texttt{README} for instructions on how
to run these tests.  As a running example, we will use the program
\texttt{asum.ys} in this subdirectory.  This program is shown as
CS:APP2e Figure \ref{fig:arch:code-yso}.  The compiled version of the
program is shown in Figure \ref{fig:sim:code-yso}.

\item[\texttt{ptest}]
Scripts that generate systematic regression tests of the different
instructions, the different jump possibilities, and different hazard
possibilities.  These scripts are very good at finding bugs in your
homework solutions.
See file \texttt{README} for instructions on how to run these tests.
\end{description}

\begin{figure}
\begin{ccode}
\begin{alltt}
{\scriptsize   1}                       | # Execution begins at address 0
{\scriptsize   2}   0x000:              |         .pos 0
{\scriptsize   3}   0x000: 30f400010000 | init:   irmovl Stack, %esp      # Set up stack pointer
{\scriptsize   4}   0x006: 30f500010000 |         irmovl Stack, %ebp      # Set up base pointer
{\scriptsize   5}   0x00c: 8024000000   |         call Main               # Execute main program
{\scriptsize   6}   0x011: 00           |         halt                    # Terminate program
{\scriptsize   7}                       |
{\scriptsize   8}                       | # Array of 4 elements
{\scriptsize   9}   0x014:              |         .align 4
{\scriptsize  10}   0x014: 0d000000     | array:  .long 0xd
{\scriptsize  11}   0x018: c0000000     |         .long 0xc0
{\scriptsize  12}   0x01c: 000b0000     |         .long 0xb00
{\scriptsize  13}   0x020: 00a00000     |         .long 0xa000
{\scriptsize  14}                       |
{\scriptsize  15}   0x024: a05f         | Main:   pushl %ebp
{\scriptsize  16}   0x026: 2045         |         rrmovl %esp,%ebp
{\scriptsize  17}   0x028: 30f004000000 |         irmovl $4,%eax
{\scriptsize  18}   0x02e: a00f         |         pushl %eax              # Push 4
{\scriptsize  19}   0x030: 30f214000000 |         irmovl array,%edx
{\scriptsize  20}   0x036: a02f         |         pushl %edx              # Push array
{\scriptsize  21}   0x038: 8042000000   |         call Sum                # Sum(array, 4)
{\scriptsize  22}   0x03d: 2054         |         rrmovl %ebp,%esp
{\scriptsize  23}   0x03f: b05f         |         popl %ebp
{\scriptsize  24}   0x041: 90           |         ret
{\scriptsize  25}                       |
{\scriptsize  26}                       |         # int Sum(int *Start, int Count)
{\scriptsize  27}   0x042: a05f         | Sum:    pushl %ebp
{\scriptsize  28}   0x044: 2045         |         rrmovl %esp,%ebp
{\scriptsize  29}   0x046: 501508000000 |         mrmovl 8(%ebp),%ecx     # ecx = Start
{\scriptsize  30}   0x04c: 50250c000000 |         mrmovl 12(%ebp),%edx    # edx = Count
{\scriptsize  31}   0x052: 6300         |         xorl %eax,%eax          # sum = 0
{\scriptsize  32}   0x054: 6222         |         andl   %edx,%edx        # Set condition codes
{\scriptsize  33}   0x056: 7378000000   |         je     End
{\scriptsize  34}   0x05b: 506100000000 | Loop:   mrmovl (%ecx),%esi      # get *Start
{\scriptsize  35}   0x061: 6060         |         addl %esi,%eax          # add to sum
{\scriptsize  36}   0x063: 30f304000000 |         irmovl $4,%ebx          #
{\scriptsize  37}   0x069: 6031         |         addl %ebx,%ecx          # Start++
{\scriptsize  38}   0x06b: 30f3ffffffff |         irmovl $-1,%ebx         #
{\scriptsize  39}   0x071: 6032         |         addl %ebx,%edx          # Count--
{\scriptsize  40}   0x073: 745b000000   |         jne    Loop             # Stop when 0
{\scriptsize  41}   0x078: 2054         | End:    rrmovl %ebp,%esp
{\scriptsize  42}   0x07a: b05f         |         popl %ebp
{\scriptsize  43}   0x07c: 90           |         ret
{\scriptsize  44}                       |
{\scriptsize  45}                       | # The stack starts here and grows to lower addresses
{\scriptsize  46}   0x100:              |         .pos 0x100
{\scriptsize  47}   0x100:              | Stack:
\end{alltt}

\end{ccode}
\mycaption{Sample object code file.}{This code is in the file \texttt{asum.yo} in the
\texttt{y86-code} subdirectory.}  
\label{fig:sim:code-yso}
\end{figure}

\section{Utility Programs}

Once installation is complete, the \texttt{misc} directory contains
two useful programs:

\begin{description}
\item[\textsc{yas}] The Y86 assembler.  This takes a Y86 assembly code
file with extension \texttt{.ys} and generates a file with
extension \texttt{.yo}.  The generated file contains an ASCII
version of the object code, such as that shown in Figure
\ref{fig:sim:code-yso} (the same program as shown in CS:APP2e Figure
\ref{fig:arch:code-yso}).  The easiest way to invoke the assembler is
to use or create assembly code files in the \texttt{y86-code}
subdirectory.  For example, to assemble the program in file
\texttt{asum.ys} in this directory, we use the command:
\begin{tty}
\unixprompt{\em make asum.yo}
\end{tty}

\item[\textsc{yis}] The Y86 instruction simulator.  This program
executes the instructions in a Y86 machine-level program according to
the instruction set definition.  For example, suppose you want to run
the program \texttt{asum.yo} from within the subdirectory
\texttt{y86-code}.  Simply run:

\begin{tty}
\unixprompt{\em ../misc/yis asum.yo}
\end{tty}

\textsc{Yis} simulates the execution of the program and then prints
 changes to any registers or memory locations on the terminal, as
 described in CS:APP2e Section \ref{sect:arch:y86-isa}.
\end{description}

\section{Processor Simulators}

For each of the three processors, SEQ, SEQ+, and PIPE, we have
provided simulators \textsc{ssim}, \textsc{ssim+}, and
\textsc{psim} respectively. Each simulator can be run in 
TTY or GUI mode:

\begin{description}
\item [TTY mode]
Uses a minimalist, terminal-oriented interface.  Prints
everything on the terminal output.  Not very convenient for debugging
but can be installed on any system and can be used for
automated testing. The default mode for all simulators.

\item [GUI mode]
Has a graphic user interface, to be described shortly.  Very helpful
for visualizing the processor activity and for debugging modified
versions of the design.  However, it requires installation of Tcl/Tk on
your system. Invoked with the \texttt{-g} command line option.

\end{description}

\subsection{Command Line Options}

You can request a number of options from the command line:
\begin{description}

\item [\texttt{-h}]  
Prints a summary of all of the command line options.

\item[\texttt{-g} ]
Run the simulator in GUI mode (default TTY mode). 

\item[\texttt{-t} ]
Runs both the processor and the ISA simulators, comparing the
resulting values of the memory, register file, and condition codes.
If no discrepancies are found, it prints the message ``ISA Check
Succeeds.''  Otherwise, it prints information about the words of the
register file or memory that differ.  This feature is very useful for
testing the processor designs.

\item[\texttt{-l m} ]
Sets the instruction limit, executing at most $m$ instructions
before halting (default 10000 instructions).

\item[\texttt{-v n} ]
Sets the verbosity level to $n$, which must be between 0 and 2 with a
default value of 2. 


\end{description}
Simulators running in GUI mode must be invoked with the name of an object
file on the command line. In TTY mode, the object file name is
optional, coming from \texttt{stdin} by default. 

Here are some typical invocations of the simulators (from the \texttt{y86-code} 
subdirectory):
\begin{tty}
\unixprompt{\em ../seq/ssim -h}
\unixprompt{\em ../seq/ssim -t < asum.yo}
\unixprompt{\em ../pipe/psim -t -g asum.yo}
\end{tty}
The first case prints a summary of the command line options for
\textsc{ssim}.  The second case runs the SEQ simulator in TTY mode,
reading object file \texttt{asum.yo} from \texttt{stdin}.  The third case runs
the PIPE simulator in GUI mode, executing the instructions object file
\texttt{asum.yo}. In both the second and third cases, the results are
compared with the results from the higher-level ISA simulator.

\subsection{SEQ and SEQ+ Simulators}

\begin{figure}
\centerline{\includegraphics*[scale=0.8]{seq-cntl.eps}}
\caption{Main control panel for SEQ simulator}
\label{fig:sim:seq-cntl}
\end{figure}

\begin{figure}
\centerline{\includegraphics*[scale=0.8]{seq-code.eps}}
\caption{Code display window for SEQ simulator}
\label{fig:sim:seq-code}
\end{figure}

\begin{figure}
\centerline{\includegraphics*[scale=1.0]{pipe-mem.eps}}
\caption{Memory display window for SEQ simulator}
\label{fig:sim:seq-mem}
\end{figure}

The GUI version of the SEQ processor simulator is invoked with 
an object code filename on the command line:
\begin{tty}
\unixprompt{\em ../seq/ssim -g asum.yo &}
\end{tty}
where the ``\texttt{\&}'' at the end of the command line allows the
simulator to run in background mode.  The simulation program starts up
and creates three windows, as illustrated in Figures
\ref{fig:sim:seq-cntl}--\ref{fig:sim:seq-mem}.

The first window (Figure \ref{fig:sim:seq-cntl}) is the main control
panel. If the HCL file was compiled by \textsc{hcl2c} with the
\texttt{-n name} option, then the title of the main control window will appear
as ``\texttt{Y86 Processor: name}''
Otherwise it will appear as simply
``\texttt{Y86 Processor}.''

The main control window contains buttons to control the simulator as
well as status information about the state of the processor.  The
different parts of the window are labeled in the figure:

\begin{description}
\item[Control:] The buttons along the top control the simulator.
Clicking the \ival{Quit} button causes the simulator to exit.
Clicking the \ival{Go} button causes the simulator to start running.
Clicking the \ival{Stop} button causes the simulator to stop
temporarily.  Clicking the \ival{Step} button causes the simulator to
execute one instruction and then stop.  Clicking the \ival{Reset}
button causes the simulator to return to its initial state, with the
program counter at address 0, the registers set to 0s, the memory
erased except for the program, the condition codes set with
$\mzf = 1$, $\mcf = 0$, and $\mof = 0$, and the program status set to .
\verb@AOK@.

The slider below the buttons controls the speed of the simulator when
it is running.  Moving it to the right makes the simulator run faster.

\item[Stage signals:] 
This part of the display shows the values of the
different processor signals during the current instruction evaluation.
These signals are almost identical to those shown in CS:APP2e Figure
\ref{fig:arch:seq-full}.  The main difference is that the simulator
displays the name of the instruction in a field labeled \ival{Instr},
rather than the numeric values of \ival{icode} and \ival{ifun}.
Similarly, all register identifiers are shown using their names,
rather  than their numeric values, with
``\texttt{----}'' indicating that no register access is required.
\item[Register file:] This section displays the values of the eight
program registers.  The register that has been updated most recently
is shown highlighted in light blue.  Register contents are not displayed
until after the first time they are set to nonzero values.

Remember that when an instruction writes to a program register, the
register file is not updated until the beginning of the next clock
cycle.  This means that you must step the simulator one more time to
see the update take place.

\item[Stat:] This shows the status of the current instruction being
executed.  The possible values are:
\begin{description}
\item[\texttt{AOK}:] No problem encountered.
\item[\texttt{ADR}:] An addressing error has occurred either trying to read an
instruction or trying to read or write data.  Addresses cannot exceed
\texttt{0x0FFF}.
\item[\texttt{INS}:] An illegal instruction was encountered.
\item[\texttt{HLT}:] A \texttt{halt} instruction was encountered.
\end{description}

\item[Condition codes:]
These show the values of the three condition codes: 
\texttt{ZF}, \texttt{SF}, and \texttt{OF}\@.

Remember that when an instruction changes the condition codes, the
condition code register is not updated until the beginning of the next clock
cycle.  This means that you must step the simulator one more time to
see the update take place.

\end{description}

The processor state illustrated in Figure \ref{fig:sim:seq-cntl} is
for the second execution of line 38 of the \texttt{asum.yo} program
shown in Figure \ref{fig:sim:code-yso}.  We can see that the program
counter is at \texttt{0x06b}, that it has processed the instruction
\texttt{addl \ebxreg{} \ecxreg{}}, that register \eaxreg{} holds
\texttt{0xcd}, the sum of the first two array elements, and \edxreg{}
holds 3, the count that is about to be decremented.  Register
\ecxreg{} holds \texttt{0x1c}, the address of the third array
element.  Register \ebxreg{} still holds the value 4 (from line 36) but there 
is a pending write of \texttt{0xFFFFFFFF} to this register
(since \ival{dstE} is set to \ebxreg{} and \ival{valE} is set
to \texttt{0xFFFFFFFF}).  This write will take place at the start of the
next clock cycle.

The window depicted in Figure \ref{fig:sim:seq-code} shows the object
code file that is being executed by the simulator.%
%%\footnote{On some windowing systems, this window begins life
%%as a closed icon if the object file is large. If this happens, 
%%simply click on the icon to expand the window.}
The edit box
identifies the file name of the program being executed.  You can edit
the file name in this window and click the \ival{Load} button to load
a new program.  The left hand side of the display shows the object
code being executed, while the right hand side shows the text from the
assembly code file.  The center has an asterisk (\texttt{*}) to
indicate which instruction is currently being simulated.  This
corresponds to line 38 of the \texttt{asum.yo} program shown in Figure
\ref{fig:sim:code-yso}.

The window shown in Figure \ref{fig:sim:seq-mem} shows the contents of
the memory.  It shows only those locations between the minimum and
maximum addresses that have changed since the program began executing.
Each row shows the contents of four memory words.  Thus, each row
shows 16 bytes of the memory, where the addresses of the bytes differ
in only their least significant hexadecimal digits.  To the left of
the memory values is the ``root'' address, where the least significant
digit is shown as ``\texttt{-}''.  Each column then corresponds to
words with least significant address digits \texttt{0x0},
\texttt{0x4}, \texttt{0x8}, and \texttt{0xc}.  The example shown in
Figure \ref{fig:sim:seq-mem} has arrows indicating memory locations \texttt{0x00e0},
\texttt{0x00e4}, \texttt{0x00e8}, and
\texttt{0x00ec}.

The memory contents illustrated in the figure show the stack contents
of the \texttt{asum.yo} program shown in Figure \ref{fig:sim:code-yso} during the
execution of the \texttt{Sum} procedure.  Looking at the stack
operations that have taken place so far, we see that \espreg{} and
\ebpreg{} were
initialized to \texttt{0x100} (lines 3 and 4).  The call to
\texttt{Main} on line 5 pushes the return pointer \texttt{0x011},
which is written to address \texttt{0x00fc}.  Procedure \texttt{Main}
starts by pushing \ebpreg{}, writing \texttt{0x100} to \texttt{0x00f8}.
It then pushes the value of \eaxreg{} (line 18), writing \texttt{0x4}
to \texttt{0x00f4} and \edxreg{} (line 20), writing \texttt{0x14} (the
address of the array) to \texttt{0x00f0}.  The call to \texttt{Sum} on
line 21 causes the return pointer \texttt{0x3d} to be written to
address \texttt{0x00ec}.  Within \texttt{Sum}, pushing \ebpreg{}
causes \texttt{0xf8} to be written to address \texttt{0x00e8}.
That accounts for all of the words shown
in this memory display, and for the stack pointer being set to \texttt{0xe8}.

\begin{figure}
\centerline{\includegraphics*[scale=0.8]{seq+-cntl.eps}}
\caption{Main control panel for SEQ+ simulator}
\label{fig:sim:seq+-cntl}
\end{figure}

Figure \ref{fig:sim:seq+-cntl} shows the control panel window for the
SEQ+ simulator, when executing the same object code file and when at
the same point in this program.  We can see that the only difference
is in the ordering of the stages and the different signals listed.
These signals correspond to those in CS:APP2e Figure
\ref{fig:arch:seq+-full}.  The SEQ+ simulator also generates code and
memory windows.  These have identical format to those for the SEQ simulator.

\subsection{PIPE Simulator}
\begin{figure}
\centerline{\includegraphics*[scale=0.8]{pipe-cntl.eps}}
\caption{Main control panel for PIPE simulator}
\label{fig:sim:pipe-cntl}
\end{figure}

\begin{figure}
\centerline{\includegraphics*[scale=1.0]{pipe-register.eps}}
\caption{View of single pipe register in control panel for PIPE
simulator}
\label{fig:sim:pipe-register}
\end{figure}

The PIPE simulator also generates three windows.  Figure
\ref{fig:sim:pipe-cntl} shows the control panel.  It has the same set
of controls, and the same display of the register file and condition
codes.  The middle section shows the state of the pipeline registers.
The different fields correspond to those in CS:APP2e Figure
\ref{fig:arch:pipe-full}. At the bottom of this panel is a display
showing the number of cycles that have been simulated (not
including the initial cycles required to get the pipeline flowing),
the number of instructions that have completed, and the resulting
CPI\@.

As illustrated in the close-up view of Figure
\ref{fig:sim:pipe-register}, each pipeline register is displayed with
two parts.  The upper values in white boxes show the current values in
the pipeline register.  The lower values with a gray background show
the inputs to pipeline register.  These will be loaded into the
register on the next clock cycle, unless the register bubbles or
stalls.

The flow of values through the PIPE simulator is quite different from
that for the SEQ or SEQ+ simulator.  With SEQ and SEQ+, the control
panel shows the values resulting from executing a single instruction.
Each step of the simulator performs one complete instruction execution.
With PIPE, the control panel shows the values for the multiple
instructions flowing through the pipeline.  Each step of the simulator
performs just one stage's worth of computation for each instruction.

\begin{figure}
\centerline{\includegraphics*[scale=0.8]{pipe-code.eps}}
\caption{Code display window for PIPE simulator}
\label{fig:sim:pipe-code}
\end{figure}

Figure \ref{fig:sim:pipe-code} shows the code display for the PIPE
simulator.  The format is similar to that for SEQ and SEQ+, except
that rather than a single marker indicating which instruction is being
executed, the display indicates which instructions are in each
state of the pipeline, using characters \texttt{F}, \texttt{D},
\texttt{E}, \texttt{M}, and \texttt{W}, for the fetch, decode,
execute, memory, and write-back stages.

The PIPE simulator also generates a window to display the memory
contents.  This has an identical format to the one shown for SEQ
(Figure \ref{fig:sim:seq-mem}).

The example shown in Figures \ref{fig:sim:pipe-cntl} and
\ref{fig:sim:pipe-code} show the status of the pipeline when executing
the loop in lines 34--40 of Figure \ref{fig:sim:code-yso}.  We can see
that the simulator has begun the second iteration of the loop.
The status of the stages is as follows:
\begin{description}
\item[Write back:] The loop-closing \texttt{jne} instruction (line 40) is finishing.

\item[Memory:] The \texttt{mrmovl} instruction (line 34) has just read
\texttt{0x0C0} from address \texttt{0x018}.  We can see the address in
\ival{valE} of pipeline register M, and the value read from
memory at the input of \ival{valM} to pipeline register W\@.

\item[Execute:] This stage contains a bubble.  The bubble was inserted
due to the load-use dependency between the \texttt{mrmovl} instruction
(line 34) and the \texttt{addl} instruction (line 35).  It can be seen
that this bubble acts like a \texttt{nop} instruction.  This explains why
there is no instruction in Figure \ref{fig:sim:pipe-code} labeled with ``E.''

\item[Decode:] The \texttt{addl} instruction (line 35) has just read
\texttt{0x00D} from register \eaxreg{}.  It also read \texttt{0x00D}
from register \esireg{}, but we can see that the forwarding logic has
instead used the value \texttt{0x0C0} that has just been read from
memory (seen as the input to \ival{valM} in pipeline register W) as
the new value of \ival{valA} (seen as the input to \ival{valA} in
pipeline register E).

\item[Fetch:] The \texttt{irmovl} instruction (line 38) has just been
fetched from address \texttt{0x063}.  The new value of the PC is
predicted to be \texttt{0x069}.
\end{description}

Associated with each stage is its status field \texttt{Stat}
This field shows the status of the instruction in that
 stage of the pipeline.  Status \texttt{AOK} means that no exception has
 been encountered.  Status value \texttt{BUB} indicates that a bubble is in this
 stage, rather than a normal instruction.  Other possible status
 values are: \texttt{ADR} when an invalid memory location is referenced,
\texttt{INS} when an illegal instruction code is encountered, \texttt{PIP} when
 a problem arose in the pipeline (this occurs when both the stall and
 the bubble signals for some pipeline register are set to 1),
and \texttt{HLT}
 when a halt instruction is encountered.  The simulator will stop
 when any of these last four cases reaches the write-back stage.

Carrying the status for an individual instruction through
 the pipeline along with the rest of the information about that
 instruction enables precise handling of the different exception
 conditions, as described in CS:APP2e Section \ref{sect:arch:pipe-exceptions}.

\section{Some Advice}

The following are some miscellaneous tips, learned from experience we
have gained in using these simulators.

\begin{itemize}
\item {\em Get familiar with the simulator operation.}  Try running
some of the example programs in the \texttt{y86-code} directory.  Make
sure you understand how each instruction gets processed for some small
examples.  Watch for interesting cases such as mispredicted branches,
load interlocks, and procedure returns.

\item {\em You need to hunt around for information.}  Seeing the
effect of data forwarding is especially tricky.  There are seven
possible sources for signal \ival{valA} in pipeline register E, and
six possible sources for signal \ival{valB}\@.  To see which one was
selected, you need to compare the input to these pipeline register
fields to the values of the possible sources.
The possible sources
are:
\begin{description}
\item[$\regref{\ival{d\_srcA}}$] The source register is identified by
the input to \ival{srcA} in pipeline register E\@.  The register contents are shown at the bottom.
\item[$\regref{\ival{d\_srcB}}$] The source register is identified by
the input to \ival{srcB} in pipeline register E\@.  The register contents are shown at the bottom.
\item[\ival{D\_valP}] This value is part of the state of pipeline register D\@.
\item[\ival{e\_valE}] This value is at the input to field \ival{valE}
in pipeline register M\@.
\item[\ival{M\_valE}] This value is part of the state of pipeline register M\@.
\item[\ival{m\_valM}] This value is at the input to field \ival{valM}
in pipeline register W.
\item[\ival{W\_valE}] This value is part of the state of pipeline register W\@.
\item[\ival{W\_valM}] This value is part of the state of pipeline register M\@.
\end{description}

\item {\em Do not overwrite your code.}  Since the data and code share
the same address space, it is easy to have a program overwrite some of
the code, causing complete chaos when it attempts to execute the
overwritten instructions.  It is important to set up the stack to be
far enough away from the code to avoid this.

\item {\em Avoid large address values.}  The simulators do not allow
any addresses greater than \texttt{0x0FFF}.  In addition, the memory
display becomes unwieldy if you modify memory locations spanning a
wide range of addresses.

\item {\em Be aware of some ``features'' of the GUI-mode simulators
(\textsc{ssim}, \textsc{ssim+}, and \textsc{psim}.)}
\begin{itemize}
\item
You must must execute the programs from their home directories.  In
other words, to run \textsc{ssim} or \textsc{ssim+}, you must be in
the \texttt{seq} directory, while you must be in the \texttt{pipe}
subdirectory to run \textsc{psim}.  This requirement arises due to the
way the Tcl interpreter locates the configuration file for
the simulator.

\item If you are running in GUI mode on a Unix box, remember to
initialize the DISPLAY environment variable:
\begin{tty}
unix> {\em setenv DISPLAY myhost.edu:0}
\end{tty}

\item With some Unix X Window managers, the ``Program Code'' window 
begins life as a closed icon. If you don't see this window when the
simulator starts, you'll need to expand the expand manually by
clicking on it.

\item With some Microsoft Windows X servers, the ``Memory Contents'' window
does not automatically resize itself when the memory contents change.
In these cases, you'll need to resize the window manually to see the
memory contents.

\item The simulators 
will terminate with a segmentation fault if you ask them to execute a file
that is not a valid Y86 object file.
\end{itemize}

\end{itemize}
\end{document}
