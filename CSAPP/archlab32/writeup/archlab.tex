\documentclass[11pt]{article}

%% Comment command
\newcommand{\comment}[1]{}

%% Two part captions.  First part is title.  Second is explanation.
\newcommand{\mycaption}[2]{\caption[#1]{\small \textsf{\textbf{#1} #2}}}

%% tty - for displaying TTY input and output
\newenvironment{tty}%
{\small\begin{alltt}}%
{\end{alltt}}

%% ccode- for displaying formatted C code (c2tex) 
\newenvironment{ccode}%
{\small}%
{}

%% codefrag - for displaying unformatted code fragments
\newenvironment{codefrag}%
{\small\begin{alltt}}%
{\end{alltt}%
}

\usepackage{times}
\usepackage{alltt}

%% Page layout
\oddsidemargin 0pt
\evensidemargin 0pt
\textheight 600pt
\textwidth 469pt
\setlength{\parindent}{0em}
\setlength{\parskip}{1ex}


\begin{document}

\title{CS 349, Summer 2011\\
Optimizing the Performance of a Pipelined Processor\\
Assigned: June~6, Due: June~21, 11:59PM}

\author{}
\date{}

\maketitle

Harry Bovik ({\tt bovik@cs.cmu.edu}) is the lead person for this
assignment.

\section{Introduction}

In this lab, you will learn about the design and implementation of a
pipelined Y86 processor, optimizing both it and a benchmark
program to maximize performance.
You are allowed to make any semantics preserving
transformations to the benchmark program, or to make enhancements to
the pipelined processor, or both. When you have completed the lab, you
will have a keen appreciation for the interactions between code and
hardware that affect the performance of your programs.

The lab is organized into three parts, each with its own handin.  In
Part A you will write some simple Y86 programs and become familiar
with the Y86 tools. In Part B, you will extend the SEQ simulator with
two new instructions. These two parts will prepare you for Part C, the
heart of the lab, where you will optimize the Y86 benchmark program and
the processor design.

\section{Logistics}

You will work on this lab alone.  

Any clarifications and revisions to the assignment will be posted on
the course Web page.

\section{Handout Instructions}

\begin{quote}
\bf SITE-SPECIFIC: Insert a paragraph here that explains how students
should download the \texttt{archlab-handout.tar} file.
\end{quote}

\begin{enumerate}
\item Start by copying the file {\tt archlab-handout.tar}
to a (protected) directory in which you plan to do your work.

\item Then give the command: \verb@tar xvf archlab-handout.tar@.  This will
cause the following files to be unpacked into the directory: {\tt
README}, {\tt Makefile}, {\tt sim.tar}, {\tt archlab.ps}, {\tt
archlab.pdf}, and {\tt simguide.pdf}.  

\item Next, give the command \verb@tar xvf sim.tar@. This will create the
directory {\tt sim}, which contains your personal copy of the Y86
tools. You will be doing all of your work inside this directory.

\item Finally, change to the {\tt sim} directory and build the
Y86 tools:
\begin{tty}
unix> {\em cd sim}
unix> {\em make clean; make}
\end{tty}
\end{enumerate}


\section{Part A}

You will be working in directory \texttt{sim/misc} in this part.

Your task is to write and simulate the following three Y86 programs.
The required behavior of these programs is defined by the example C
functions in \texttt{examples.c}. Be sure to put your name and ID in a
comment at the beginning of each program.  You can test your programs by first 
assembling them with the program \textsc{yas} and then running them with the instruction set simulator \textsc{yis}.

In all of your Y86 functions, you should follow the IA32 conventions
for the structure of the stack frame and for register usage
instructions, including saving and restoring any callee-save registers
that you use.

\subsection*{\texttt{sum.ys}: Iteratively sum linked list elements}

Write a Y86 program \texttt{sum.ys} that iteratively sums the
elements of a linked list. Your program should consist of some code
that sets up the stack structure, invokes a function, and then halts.
In this case, the function should be
Y86 code for a function (\texttt{sum\_list}) 
that is functionally equivalent to the C \texttt{sum\_list} function in
Figure~\ref{fig:examples}. Test your program using the following
three-element list:

\begin{codefrag}
# Sample linked list
.align 4
ele1:
        .long 0x00a
        .long ele2
ele2:
        .long 0x0b0
        .long ele3
ele3:
        .long 0xc00
        .long 0
\end{codefrag} 

\begin{figure}
\begin{ccode}
\begin{alltt}
{\scriptsize   1} /* linked list element */
{\scriptsize   2} typedef struct ELE \verb:{:
{\scriptsize   3}     int val;
{\scriptsize   4}     struct ELE *next;
{\scriptsize   5} \verb:}: *list_ptr;
{\scriptsize   6} 
{\scriptsize   7} /* sum_list - Sum the elements of a linked list */
{\scriptsize   8} int sum_list(list_ptr ls)
{\scriptsize   9} \verb:{:
{\scriptsize  10}     int val = 0;
{\scriptsize  11}     while (ls) \verb:{:
{\scriptsize  12}         val += ls->val;
{\scriptsize  13}         ls = ls->next;
{\scriptsize  14}     \verb:}:
{\scriptsize  15}     return val;
{\scriptsize  16} \verb:}:
{\scriptsize  17} 
{\scriptsize  18} /* rsum_list - Recursive version of sum_list */
{\scriptsize  19} int rsum_list(list_ptr ls)
{\scriptsize  20} \verb:{:
{\scriptsize  21}     if (!ls)
{\scriptsize  22}         return 0;
{\scriptsize  23}     else \verb:{:
{\scriptsize  24}         int val = ls->val;
{\scriptsize  25}         int rest = rsum_list(ls->next);
{\scriptsize  26}         return val + rest;
{\scriptsize  27}     \verb:}:
{\scriptsize  28} \verb:}:
{\scriptsize  29} 
{\scriptsize  30} /* copy_block - Copy src to dest and return xor checksum of src */
{\scriptsize  31} int copy_block(int *src, int *dest, int len)
{\scriptsize  32} \verb:{:
{\scriptsize  33}     int result = 0;
{\scriptsize  34}     while (len > 0) \verb:{:
{\scriptsize  35}         int val = *src++;
{\scriptsize  36}         *dest++ = val;
{\scriptsize  37}         result ^= val;
{\scriptsize  38}         len--;
{\scriptsize  39}     \verb:}:
{\scriptsize  40}     return result;
{\scriptsize  41} \verb:}:
\end{alltt}

\end{ccode}
\mycaption{C versions of the Y86 solution functions.}
{See \texttt{sim/misc/examples.c}}
\label{fig:examples}
\end{figure}

\subsection*{\texttt{rsum.ys}: Recursively sum linked list elements}

Write a Y86 program 
\texttt{rsum.ys} that recursively sums the elements of a linked
list.  This code should be similar to the code in \texttt{sum.ys},
except that it should use a function \texttt{rsum\_list} that
recursively sums a list of numbers, as shown with the C function
\texttt{rsum\_list} in 
Figure~\ref{fig:examples}.  Test your
program using the same three-element list you used for testing
\texttt{list.ys}.

\subsection*{\texttt{copy.ys}: Copy a source block to a destination block}

Write a program (\texttt{copy.ys}) that copies a block of words
from one part of memory to another (non-overlapping area) area of
memory, computing the checksum (Xor) of all the words copied.

Your program should consist of code that sets up a stack frame,
invokes a function \texttt{copy\_block}, and then halts.  The function
should be functionally equivalent to the C function
\texttt{copy\_block} shown in Figure
Figure~\ref{fig:examples}.  Test your
program using the following three-element source and destination
blocks:

\begin{codefrag}
.align 4
# Source block
src:
        .long 0x00a
        .long 0x0b0
        .long 0xc00

# Destination block
dest:
        .long 0x111
        .long 0x222
        .long 0x333
\end{codefrag}

\section{Part B}

You will be working in directory \texttt{sim/seq} in this part.

Your task in Part B is to extend the SEQ processor to support two new
instructions:
\texttt{iaddl} (described in Homework problems 4.47 and 4.49)\footnote{In the international edition, \texttt{iaddl} is described in problems 4.48 and 4.50}. 
\texttt{leave} (described in Homework problems 4.48 and 4.50)\footnote{In the international edition, \texttt{leave} is described in problems 4.47 and 4.49}.
To add these instructions, you will modify the file
\texttt{seq-full.hcl},
which implements the version of SEQ described in the CS:APP2e textbook.
In addition, it contains declarations of some constants that you will
need for your solution.

Your HCL file must begin with a header comment containing
the following information:
\begin{itemize}
\item Your name and ID.
\item A description of the computations required for the \texttt{iaddl}
instruction. Use the descriptions of \texttt{irmovl} and {\tt OPl} 
in Figure 4.18 in the CS:APP2e text as a guide.
\item A description of the computations required for the \texttt{leave}
instruction. Use the description of \texttt{popl} in Figure 4.20 in
the CS:APP2e text as a guide.
\end{itemize}

\subsection*{Building and Testing Your Solution}

Once you have finished modifying the \texttt{seq-full.hcl} file, then
you will need to build a new instance of the SEQ simulator (\texttt{ssim})
based on this HCL file, and then test it:

\begin{itemize}
\item {\em Building a new simulator.} 
You can use \texttt{make} to build a new SEQ simulator:
\begin{codefrag}
unix> {\em make VERSION=full}
\end{codefrag}
This builds a version of \texttt{ssim} that uses the control logic you
specified in \texttt{seq-full.hcl}.  To save typing, you can assign
\texttt{VERSION=full} in the Makefile. 

\item {\em Testing your solution on a simple Y86 program.}
For your initial testing, we recommend running simple programs such
as \texttt{asumi.yo} (testing \texttt{iaddl}) and \texttt{asuml.yo}
(testing \texttt{leave}) in TTY mode, 
comparing the results against the ISA simulation:
\begin{tty}
unix> {\em ./ssim -t ../y86-code/asumi.yo}
unix> {\em ./ssim -t ../y86-code/asuml.yo}
\end{tty}
If the ISA test fails, then you should debug your implementation by
single stepping the simulator in GUI mode:
\begin{tty}
unix> {\em ./ssim -g ../y86-code/asumi.yo}
unix> {\em ./ssim -g ../y86-code/asuml.yo}
\end{tty}

\item {\em Retesting your solution using the benchmark programs.}
Once your simulator is able to correctly execute small programs, then
you can automatically test it on the Y86 benchmark programs in {\tt
../y86-code}:
\begin{tty}
unix> {\em (cd ../y86-code; make testssim)}
\end{tty}
This will run {\tt ssim} on the benchmark programs and check for
correctness by comparing the resulting processor state with the state
from a high-level ISA simulation.
Note that none of these programs test the added instructions.
You are simply making sure that your solution did not inject errors
for the original instructions.
See file
\texttt{../y86-code/README} file for more details.

\item {\em Performing regression tests.}
Once you can execute the benchmark programs correctly, then you should
run the extensive set of regression tests in {\tt ../ptest}.  To test
everything except {\tt iaddl} and {\tt leave}:
\begin{tty}
unix> {\em (cd ../ptest; make SIM=../seq/ssim)}
\end{tty}
To test your implementation of {\tt iaddl}:
\begin{tty}
unix> {\em (cd ../ptest; make SIM=../seq/ssim TFLAGS=-i)}
\end{tty}
To test your implementation of {\tt leave}:
\begin{tty}
unix> {\em (cd ../ptest; make SIM=../seq/ssim TFLAGS=-l)}
\end{tty}
To test both {\tt iaddl} and {\tt leave}:
\begin{tty}
unix> {\em (cd ../ptest; make SIM=../seq/ssim TFLAGS=-il)}
\end{tty}
\end{itemize}

For more information on the SEQ simulator refer to the handout
\textit{CS:APP2e Guide to Y86 Processor Simulators} (\texttt{simguide.pdf}).

\section{Part C}

You will be working in directory \texttt{sim/pipe} in this part.

The \texttt{ncopy} function in Figure~\ref{fig:ncopyc} copies a
\texttt{len}-element integer array \texttt{src} to a non-overlapping
\texttt{dst}, returning a count of the number of positive integers
contained in \texttt{src}.
\begin{figure}
\begin{ccode}
\begin{alltt}
{\scriptsize   1} /*
{\scriptsize   2}  * ncopy - copy src to dst, returning number of positive ints
{\scriptsize   3}  * contained in src array.
{\scriptsize   4}  */
{\scriptsize   5} int ncopy(int *src, int *dst, int len)
{\scriptsize   6} \verb:{:
{\scriptsize   7}     int count = 0;
{\scriptsize   8}     int val;
{\scriptsize   9} 
{\scriptsize  10}     while (len > 0) \verb:{:
{\scriptsize  11}         val = *src++;
{\scriptsize  12}         *dst++ = val;
{\scriptsize  13}         if (val > 0)
{\scriptsize  14}             count++;
{\scriptsize  15}         len--;
{\scriptsize  16}     \verb:}:
{\scriptsize  17}     return count;
{\scriptsize  18} \verb:}:
\end{alltt}

\end{ccode}
\mycaption{C version of the \texttt{ncopy} function.}{See \texttt{sim/pipe/ncopy.c.}}
\label{fig:ncopyc}
\end{figure}
Figure~\ref{fig:ncopys} shows the baseline Y86 version of \texttt{ncopy}. 
\begin{figure}
\begin{ccode}
\begin{alltt}
{\scriptsize   1} ##################################################################
{\scriptsize   2} # ncopy.ys - Copy a src block of len ints to dst.
{\scriptsize   3} # Return the number of positive ints (>0) contained in src.
{\scriptsize   4} #
{\scriptsize   5} # Include your name and ID here.
{\scriptsize   6} #
{\scriptsize   7} # Describe how and why you modified the baseline code.
{\scriptsize   8} #
{\scriptsize   9} ##################################################################
{\scriptsize  10} # Do not modify this portion
{\scriptsize  11} # Function prologue.
{\scriptsize  12} ncopy:  pushl %ebp              # Save old frame pointer
{\scriptsize  13}         rrmovl %esp,%ebp        # Set up new frame pointer
{\scriptsize  14}         pushl %esi              # Save callee-save regs
{\scriptsize  15}         pushl %ebx
{\scriptsize  16}         pushl %edi
{\scriptsize  17}         mrmovl 8(%ebp),%ebx     # src
{\scriptsize  18}         mrmovl 16(%ebp),%edx    # len
{\scriptsize  19}         mrmovl 12(%ebp),%ecx    # dst
{\scriptsize  20} 
{\scriptsize  21} ##################################################################
{\scriptsize  22} # You can modify this portion
{\scriptsize  23}         # Loop header
{\scriptsize  24}         xorl %eax,%eax          # count = 0;
{\scriptsize  25}         andl %edx,%edx          # len <= 0?
{\scriptsize  26}         jle Done                # if so, goto Done:
{\scriptsize  27} 
{\scriptsize  28} Loop:   mrmovl (%ebx), %esi     # read val from src...
{\scriptsize  29}         rmmovl %esi, (%ecx)     # ...and store it to dst
{\scriptsize  30}         andl %esi, %esi         # val <= 0?
{\scriptsize  31}         jle Npos                # if so, goto Npos:
{\scriptsize  32}         irmovl $1, %edi
{\scriptsize  33}         addl %edi, %eax         # count++
{\scriptsize  34} Npos:   irmovl $1, %edi
{\scriptsize  35}         subl %edi, %edx         # len--
{\scriptsize  36}         irmovl $4, %edi
{\scriptsize  37}         addl %edi, %ebx         # src++
{\scriptsize  38}         addl %edi, %ecx         # dst++
{\scriptsize  39}         andl %edx,%edx          # len > 0?
{\scriptsize  40}         jg Loop                 # if so, goto Loop:
{\scriptsize  41} ##################################################################
{\scriptsize  42} # Do not modify the following section of code
{\scriptsize  43} # Function epilogue.
{\scriptsize  44} Done:
{\scriptsize  45}         popl %edi               # Restore callee-save registers
{\scriptsize  46}         popl %ebx
{\scriptsize  47}         popl %esi
{\scriptsize  48}         rrmovl %ebp, %esp
{\scriptsize  49}         popl %ebp
{\scriptsize  50}         ret
{\scriptsize  51} ##################################################################
{\scriptsize  52} # Keep the following label at the end of your function
{\scriptsize  53} End:
\end{alltt}

\end{ccode}
\mycaption{Baseline Y86 version of the \texttt{ncopy} function.}{See \texttt{sim/pipe/ncopy.ys.}}
\label{fig:ncopys}
\end{figure}
The file \texttt{pipe-full.hcl} contains a copy of the HCL
code for PIPE, along with a declaration of the constant value \texttt{IIADDL}.

Your task in Part C is to modify \texttt{ncopy.ys} and
\texttt{pipe-full.hcl} with the goal of making \texttt{ncopy.ys} run
as fast as possible.

You will be handing in two files: \texttt{pipe-full.hcl} and 
\texttt{ncopy.ys}. Each file should begin with a header comment
with the following information:
\begin{itemize}
\item Your name and ID.
\item A high-level description of your code. In each case, describe
how and why you modified your code.
\end{itemize}

\subsection*{Coding Rules}

You are free to make any modifications you wish, with the following
constraints:

\begin{itemize}
\item Your \texttt{ncopy.ys} function must work for arbitrary array
sizes. You might be tempted to hardwire your solution for 64-element
arrays by simply coding 64 copy instructions, but this would be a bad
idea because we will be grading your solution based on its performance
on arbitrary arrays.

\item Your \texttt{ncopy.ys} function must run correctly with \textsc{yis}.
By correctly, we mean that it must correctly copy the \texttt{src} block
{\em and} return (in \texttt{\%eax}) the correct number of positive integers.

\item The assembled version of your \texttt{ncopy} file must not be
more than 1000 bytes long.  You can check the length of any program
with the \texttt{ncopy} function embedded using the provided script
\texttt{check-len.pl}:
\begin{tty}
unix> {\em ./check-len.pl < ncopy.yo}
\end{tty}

\item Your \texttt{pipe-full.hcl} implementation must pass the 
regression tests in \texttt{../y86-code} and \texttt{../ptest}
(without the \texttt{-il} flags that test {\tt iaddl} and {\tt
leave}).
\end{itemize}

Other than that, you are free to implement the {\tt iaddl} instruction
if you think that will help.  You may make any semantics preserving transformations to
the {\tt ncopy.ys} function, such as reordering instructions, replacing
groups of instructions with single instructions, deleting some
instructions, and adding other instructions.
You may find it useful to read about loop unrolling in Section 5.8 of CS:APP2e.

\subsection*{Building and Running Your Solution}

In order to test your solution, you will need to build a driver
program that calls your \texttt{ncopy} function.  We have provided you
with the \texttt{gen-driver.pl} program that generates a driver
program for arbitrary sized input arrays. For example, typing

\begin{tty}
unix> {\em make drivers}
\end{tty}
will construct the following two useful driver programs:

\begin{itemize}
\item \texttt{sdriver.yo}: A {\em small driver program} that tests
an \texttt{ncopy} function on small arrays with 4 elements.  
If your solution is correct, then this program will halt with a value
of 2 in register \texttt{\%eax} after copying the \texttt{src}
array.

\item \texttt{ldriver.yo}: A {\em large driver program} that tests
an \texttt{ncopy} function on larger arrays with 63 elements.  If
your solution is correct, then this program will halt with a value of
31 (\texttt{0x1f}) in register \texttt{\%eax} after copying the
\texttt{src} array.
\end{itemize}
Each time you modify your \texttt{ncopy.ys} program, you can 
rebuild the driver programs by typing
\begin{tty}
unix> {\em make drivers}
\end{tty}
Each time you modify your \texttt{pipe-full.hcl} file, you 
can rebuild the simulator by typing 
\begin{tty}
unix> {\em make psim VERSION=full}
\end{tty}
If you want to rebuild the simulator and the driver programs, type
\begin{tty}
unix> {\em make VERSION=full}
\end{tty}

To test your solution in GUI mode on a small 4-element array, type
\begin{codefrag}
unix> {\em ./psim -g sdriver.yo}
\end{codefrag}
To test your solution on a larger 63-element array, type
\begin{codefrag}
unix> {\em ./psim -g ldriver.yo}
\end{codefrag}
Once your simulator correctly runs your version of \texttt{ncopy.ys} on 
these two block lengths,
you will want to perform the following additional tests:
\begin{itemize}
\item {\em Testing your driver files on the ISA simulator.} Make 
sure that your \texttt{ncopy.ys} function works properly with
\textsc{yis}:
\begin{codefrag}
unix> {\em make drivers}
unix> {\em ../misc/yis sdriver.yo}
\end{codefrag}

\item
{\em Testing your code on a range of block lengths with the ISA
simulator.}  The Perl script \texttt{correctness.pl} generates driver
files with block lengths from 0 up to some limit (default 65), plus
some larger sizes.  It
simulates them (by default with \textsc{yis}), and checks the results.  It
generates a report showing the status for each block length:
\begin{tty}
unix> {\em ./correctness.pl}
\end{tty}
This script generates test programs where the result count varies
randomly from one run to another, and so it provides a more stringent
test than the standard drivers.

If you get incorrect results for some length $K$, you can generate a
driver file for that length that includes checking code, and where the
result varies randomly:
\begin{tty}
unix> {\em ./gen-driver.pl -f ncopy.ys -n \(K\) -rc > driver.ys}
unix> {\em make driver.yo}
unix> {\em ../misc/yis driver.yo}
\end{tty}
The program will end with register \verb@%eax@ having the following
value:
\begin{description}
\item[\texttt{0xaaaa}]: All tests pass.
\item[\texttt{0xbbbb}]: Incorrect count
\item[\texttt{0xcccc}]: Function ncopy is more than 1000 bytes long.
\item[\texttt{0xdddd}]: Some of the source data was not copied to its
destination.
\item[\texttt{0xeeee}]: Some word just before or just after the
destination region was corrupted.
\end{description}

\item {\em Testing your pipeline simulator on the benchmark programs.}
Once your simulator is able to correctly execute \texttt{sdriver.ys}
and \texttt{ldriver.ys},
you should test it against the Y86 benchmark programs in 
{\tt ../y86-code}:
\begin{tty}
unix> {\em (cd ../y86-code; make testpsim)}
\end{tty}
This will run {\tt psim} on the benchmark programs and compare 
results with \textsc{yis}.


\item {\em Testing your pipeline simulator with extensive regression tests.}
Once you can execute the benchmark programs correctly, then you should
check it with the regression tests in {\tt ../ptest}. For example,
if your solution implements the \texttt{iaddl} instruction, then
\begin{tty}
unix> {\em (cd ../ptest; make SIM=../pipe/psim TFLAGS=-i)}
\end{tty}

\item {\em Testing your code on a range of block lengths with the pipeline simulator.}
Finally, you can run the same code tests on the pipeline simulator that you did earlier with the ISA simulator
\begin{tty}
unix> {\em ./correctness.pl -p}
\end{tty}
\end{itemize}
\section{Evaluation}


The lab is worth 190 points: 30 points for Part A, 60 points for Part B,
and 100 points for Part C.

\subsection*{Part A}
Part A is worth 30 points, 10 points for each Y86 solution
program. Each solution program will be evaluated for correctness,
including proper handling of the stack and registers,
as well as functional equivalence with the example C functions in
\texttt{examples.c}.

The programs \texttt{sum.ys} and \texttt{rsum.ys} will be considered correct
if the graders do not spot any errors in them, and their respective \texttt{sum\_list} and \texttt{rsum\_list} 
functions return the sum \texttt{0xcba} in register \texttt{\%eax}.


The program \texttt{copy.ys} will be considered correct if 
the graders do not spot any errors in them, 
and the
\texttt{copy\_block} 
function returns the sum \texttt{0xcba} in register \texttt{\%eax},
copies the three words \texttt{0x00a}, \texttt{0x0b}, and
\texttt{0xc} to the 12 contiguous memory locations beginning at 
address {\tt dest}, and does not corrupt other memory locations.

\subsection*{Part B}

This part of the lab is worth 60 points: 
\begin{itemize}
\item 10 points for your description of the computations required 
for the {\tt iaddl} instruction.
\item 10 points for your description of the computations required
for the {\tt leave} instruction.
\item 10 points for passing the benchmark regression tests in
\texttt{y86-code}, to verify that your simulator still correctly
executes the benchmark suite.
\item 15 points for passing the regression tests in {\tt ptest} 
for {\tt iaddl}.
\item 15 points for passing the regression tests in {\tt ptest}
for {\tt leave}.
\end{itemize}

\subsection*{Part C}
This part of the Lab is worth 100 points: 
{\bf You will not receive any credit if either your code 
for {\tt ncopy.ys} or your 
modified simulator fails any of the tests described earlier.}
\begin{itemize}
\item 20 points each for 
your descriptions in the headers of \texttt{ncopy.ys} and 
\texttt{pipe-full.hcl} and the quality of these implementations.

\item 60 points for performance. To receive credit here, your
solution must be correct, as defined earlier. That is, \texttt{ncopy}
runs correctly with \textsc{yis}, and \texttt{pipe-full.hcl} passes
all tests in {\tt y86-code} and \texttt{ptest}.

We will express the performance of your function in units of {\em
cycles per element} (CPE).  That is, if the simulated code requires
$C$ cycles to copy a block of $N$ elements, then the CPE is $C/N$.
The PIPE simulator displays the total number of cycles required to
complete the program.  The baseline version of the
\texttt{ncopy} function running on the standard PIPE simulator with a
large 63-element array requires 914 cycles to copy 63 elements, for a
CPE of $914/63 = 14.51$.

Since some cycles are used to set up the call to \texttt{ncopy} and to
set up the loop within \texttt{ncopy}, you will find that you will get
different values of the CPE for different block lengths (generally the
CPE will drop as $N$ increases).  We will therefore evaluate the
performance of your function by computing the average of the CPEs for
blocks ranging from 1 to 64 elements.  You can use the Perl script
\texttt{benchmark.pl} in the {\tt pipe} directory
to run simulations of your {\tt ncopy.ys}
code over a range of block lengths and compute the average CPE\@.
Simply run the command
\begin{codefrag}
unix> {\em ./benchmark.pl}
\end{codefrag}
to see what happens.  For example, the baseline version of the
\texttt{ncopy} function has CPE values ranging between $46.0$ and
$14.51$, with an average of $16.44$.  Note that this Perl script does
not check for the correctness of the answer.  Use the script
\texttt{correctness.pl} for this:
\begin{codefrag}
unix> {\em ./benchmark.pl -p}
\end{codefrag}
 

You should be able to achieve an average CPE of less than $10.0$.
Our best version averages $9.27$.
If your average CPE is $c$, then your score $S$ for this portion of the
lab will be:
\begin{eqnarray*}
S & = & \left \{ \begin{array}{ll}
0\,, &  c > 12.5 \\
24.0 \cdot (12.5 - c)\,, & 10.0 \leq c \leq 12.5 \\
60\,, &  c < 10.0 \\
\end{array}
\right .
\end{eqnarray*}

By default, {\tt benchmark.pl} and {\tt correctness.pl} compile and
test {\tt ncopy.ys}. Use the {\tt -f} argument to specify a different
file name.  The {\tt -h} flag gives a complete list of the command
line arguments.

\end{itemize}


\section{Handin Instructions}

\begin{quote}
\bf SITE-SPECIFIC: Insert a description that explains how students
should hand in the three parts of the lab.  Here is the description
we use at CMU.
\end{quote}

\begin{itemize}
\item You will be handing in three sets of files:
\begin{itemize}
\item Part A: \texttt{sum.ys}, \texttt{rsum.ys}, and \texttt{copy.ys}. 
\item Part B: \texttt{seq-full.hcl}.
\item Part C: \texttt{ncopy.ys} and \texttt{pipe-full.hcl}.
\end{itemize}

\item Make sure you have included your name and ID in a comment at
the top of each of your handin files.

\item To handin your files for part X, go to your \texttt{archlab-handout}
directory and type:
\begin{tty}
    unix> {\em make handin-partX TEAM=teamname}
\end{tty}
where \texttt{X} is \texttt{a}, \texttt{b}, or \texttt{c}, and where
\texttt{teamname} is your  ID. For example, to handin Part A:
\begin{tty}
    unix> {\em make handin-parta TEAM=teamname}
\end{tty}


\item After the handin, if you discover a mistake and want to
submit a revised copy, type
\begin{tty}
    unix {\em make handin-partX TEAM=teamname VERSION=2}
\end{tty}
Keep incrementing the version number with each submission.

\item You can verify your handin by looking in 
\begin{codefrag}
    CLASSDIR/archlab/handin-partX
\end{codefrag}
You have list and insert permissions in this directory, but no
read or write permissions.
\end{itemize}


\section{Hints}

\begin{itemize}

\item By design, both \texttt{sdriver.yo} and \texttt{ldriver.yo}
are small enough to debug with in GUI mode. We find it
easiest to debug in GUI mode, and suggest that you use it.

\item If you running in GUI mode on a Unix server, make sure
that you have initialized the DISPLAY environment variable:
\begin{tty}
    unix> {\em setenv DISPLAY myhost.edu:0}
\end{tty}

\item With some X servers, the ``Program Code'' window begins life
as a closed icon when you run \texttt{psim} or \texttt{ssim} 
in GUI mode. Simply click on the icon to expand the window.

\item With some Microsoft 
Windows-based X servers, the ``Memory Contents'' window
will not automatically resize itself. You'll need to resize the window
by hand.

\item The \texttt{psim} and \texttt{ssim} simulators 
terminate with a segmentation fault if you ask them to execute a file
that is not a valid Y86 object file.

\end{itemize}

\end{document}



